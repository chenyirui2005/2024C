\section{数据预处理与分析}


在构建优化模型之前,对原始数据进行预处理和分析是保证模型准确性和有效性的基础。本章对附件中提供的乡村耕地信息、作物种植特性及2023年相关统计数据进行清洗、整理和分析,为后续种植策略优化提供可靠的数据支撑。处理流程包括数据清洗与缺失值补全、土地资源分布统计、作物适宜性分析以及经济效益评估,其整体框架如图\ref{fig:data_process}所示。


% \begin{figure}[htbp]
%     \centering
%     \includegraphics[width=0.8\textwidth]{figures/data_process.png}
%     \caption{数据预处理与分析框架}
%     \label{fig:data_process}
% \end{figure}

\subsubsection{数据清洗与补全}

数据预处理的首要目标是构建完整且准确的2023年基准数据集。在整理过程中,我们发现智慧大棚第一季度的种植记录缺失。根据附件2中说明“2023年智慧大棚第一季度的数据与普通大棚相同”,我们采用确定性参照填充方法,将普通大棚同期的完整种植数据用于填充智慧大棚缺失字段,从而形成结构完整的数据集。

获得完整数据后,我们对其进行了全面质量评估以确保准确性。针对亩产量、种植成本及销售价格等数值型变量,我们计算了其均值、标准差与值域范围等描述性统计量,用以审查是否存在超出农业生产与市场常规的极端值。对于地块编号、作物类型等分类型变量,我们通过生成其频率分布表,以检验记录的一致性与有效性。评估结果显示,所有数据均在合理区间内,无异常值或逻辑错误,证明该数据集具有高度完整性和可靠性,可直接用于后续建模分析。


