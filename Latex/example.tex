% \documentclass{cumcmthesis}
\documentclass[withoutpreface,bwprint]{cumcmthesis} %去掉封面与编号页,电子版提交的时候使用。


\usepackage[noend]{algpseudocode}
\usepackage{algorithm}
\usepackage{algpseudocode}
\usepackage{amsmath}
\usepackage[framemethod=TikZ]{mdframed}
\usepackage{url}   % 网页链接
\usepackage{subcaption} % 子标题
\usepackage{threeparttable} 
\title{面向不确定性与系统复杂性的农作物种植策略优化研究}
\tihao{A}
\baominghao{4321}
\schoolname{XX大学}
\membera{ }
\memberb{ }
\memberc{ }
\supervisor{ }%辅导老师
\yearinput{2023}
\monthinput{9}
\dayinput{8}

\setcounter{tocdepth}{2}

\begin{document}

\maketitle
\begin{abstract}
在乡村现代化与土地资源高效利用的背景下,本研究聚焦于农业生产中的资源配置优化问题。针对华北某山区乡村,旨在制定一个覆盖2024至2030年的最优农作物种植方案。该乡村的生产决策面临土地异质性、多重农艺规则以及田间管理便利性等一系列硬性约束,构成了一个复杂的约束优化环境。为应对此挑战,本文通过从\textbf{静态确定性规划、风险规避下的鲁棒优化到动态仿真优化}的递进式模型体系,旨在为该乡村提供兼顾经济效益最大化与风险稳健性的数据驱动型决策支持。

针对问题一,我们在所有经济与生产参数保持不变的确定性假设下,构建了一个\textbf{混合整数线性规划(MILP)模型}。该模型遵循\textbf{微观经济学}中\textbf{厂商利润最大化原则},在给定的生产技术与资源下,求解最优生产组合。我们分别探讨了当边际产出超出市场预期时,产品滞销或以折价销售的两种情景。鉴于该组合优化问题的计算复杂性,我们设计并实现了一种\textbf{修复式遗传算法}进行求解。结果表明,在滞销情景下,七年最优总利润为2987.65万元;在降价出售情景下,该值可提升至4191.35万元。


针对问题二,本研究在原有确定性优化模型的基础上引入了市场与生产环境的不确定性,建立了一个\textbf{鲁棒优化模型}。该模型以年度平均最低利润为目标函数,在预设的不确定性扰动下,保证种植方案在最坏情景下仍具备可行性。模型参数包括亩产量、销售量和销售价格等关键不确定变量,并通过\textbf{多种群遗传算法(MPGA)}进行求解。为验证方案的有效性,本研究将鲁棒优化方案与确定性方案在相同的随机情景下进行了对比。模拟结果表明,鲁棒优化方案不仅提升了期望利润,从40.53百万元提升至45.83百万元,还显著降低了收益波动风险,其利润标准差由5.52百万元降至3.97百万元),体现出在不确定环境下更优的稳定性与适应性。


针对问题三,本研究在模型中引入了\textbf{市场动态反馈机制},用于描述供给变化对市场均衡价格和要素成本的影响。在此基础上,构建了一个基于\textbf{价格与成本敏感度系数的仿真优化模型},以分析乡村作为市场参与者时,其决策对自身经济环境的反作用。模型的求解仍采用多种群遗传算法,并在集成\textbf{蒙特卡洛模拟}的适应度评估框架下进行计算。结果表明,得到的自适应种植策略在七年周期内的平均总利润为3902.05万元。进一步的收益分布分析显示,该方案具有较高的稳定性,在95\%的情况下总利润不低于3865.18万元。

最后,为检验模型体系的稳健性,我们进行了多维度的灵敏度分析。我们评估了管理便利性约束(分散度参数$p, q$)、核心生产要素(土地总面积)、决策者风险偏好以及市场价格反馈强度等关键参数的变动对各模型最优解的影响。分析结果证实,本研究提出的种植策略对关键参数的扰动表现出良好的稳定性,所构建的模型框架具有可靠性。






\keywords{混合整数线性规划 \quad 鲁棒优化 \quad 仿真优化 \quad 修复式遗传算法 \quad 蒙特卡洛模拟 \quad 微观经济学 \quad 市场动态反馈机制}


\end{abstract}








\section{问题背景}

在乡村振兴与农业现代化背景下,科学规划并高效利用有限土地,对保障区域粮食安全和促进乡村经济可持续发展至关重要。该问题涉及资源配置的运筹学建模,同时关联生态经济中的经济—生态系统相互作用和区域经济的内生增长机制。作物生产与品种选择受地域、气候和土壤等自然条件约束;其经济效益则依赖于市场需求、生产成本与销售价格等动态变量。基于此,需要制定兼顾经济产出最大化、生态稳定性与风险应对能力的长期种植策略\upcite{GZZP202402041}。

本研究以华北某山区乡村为对象。该乡村兼具露天耕地与设施大棚等多类型土地资源,并受轮作制度、豆类种植比例及田间管理可及性等多重约束。上述现实因素构成了一个多维约束的决策环境。为此,本文采用系统的数学建模方法,构建覆盖2024至2030年的最优种植方案,以期为土地高效利用与经济稳健增长提供数据驱动的决策支持。


\section{问题重述}

问题一:在所有经济与生产参数取基准值且保持不变的条件下,针对 2024–2030 年规划期,求解最优种植结构以最大化七年累计利润。分别在两种需求情景下求解:(i)产量超出预期的部分无法销售;(ii)超出部分以基准价格的 50\% 出售。

问题二:在问题一框架上引入参数不确定性。考虑预期销售量、单位面积产量、单位面积种植成本和销售价格在给定区间内的波动,设计一个在整个规划期内保持不变的单一种植方案,以收益—风险权衡为目标,提升方案鲁棒性,并在不利市场与生产情景下保持经济绩效稳定。

问题三:在问题二基础上纳入作物间的替代性与互补性,以及预期销售量、销售价格与种植成本的相关结构。基于这些关系的建模与放好着呢,求解可随市场参数变化更新的自适应种植策略,并与问题二的鲁棒方案进行对比评估。


\section{问题分析}

对于问题一,其在确定性前提下提出:研究对象为多周期、多地块、多品类的资源分配问题,且假定所有生产与市场参数为已知且恒定,目标为在规划期内实现累计利润最大化。该设定与古典微观经济学中厂商理论的理性与利润最大化假设一致,因此在既定资源与农艺约束下的最优决策过程可被视为对单一目标决策主体的数学建模。由此,该问题可形式化为一个混合整数线性规划(MILP)模型,以便在约束条件下求解最优种植策略。

对于问题二,其本质上是将确定性优化推广到不确定情形:预期销量、单位产量、单位种植成本与销售价格作为在预设区间内波动的有界不确定量,且题设未给出其概率分布。因缺乏分布信息,基于概率分布的随机规划方法不可行,因此应采用仅依赖波动边界的建模策略,即鲁棒优化。鲁棒优化通过最大化最坏情形下的性能以体现风险规避倾向,相较于以期望收益为目标的方法更侧重最低收益保障与下行风险控制,从而在数据有限的条件下生成更为审慎且性能稳定的种植方案。

对于问题三,其在模型中引入作物间的替代性与互补性以及价格—需求反馈等系统性相互作用。由于这些相互作用产生显著的动态非线性,模型目标函数无法用封闭的解析表达式精确表示,故基于显式解析模型的优化方法难以适用。因而应将该问题视为黑箱优化问题,并采用仿真—优化框架:首先构建能够再现关键市场与生产动态的仿真器;其次采用不依赖问题解析形式的全局搜索算法在方案空间中寻优;最后通过仿真对所得策略的性能与稳健性进行评估。我们最终的框架如图\ref{fig:all}所示。


\begin{figure}[htbp]
	\centering
	\includegraphics[width=\textwidth]{figs/1前置/全文框架}
	\caption{整体框架}
	\label{fig:all}
\end{figure}


\section{模型假设}


为确保数学模型的有效性与针对性,并明确其适用边界,我们提出以下四条基本假设:

\begin{enumerate}
    \item 模型所使用的全部历史数据均视为准确可靠。2023年的产销数据被视为代表了市场的均衡状态。

    \item 该乡村在农产品市场中被视为价格接受者。其自身的任何生产决策所导致的供给量变化,均不足以对该作物的市场结算价格产生实质性影响。

    \item 模型的经济效益评估聚焦于种植环节。所有计划销售的农产品,无论是正常价格或降价销售,均假定在当季完成交易。模型不包含仓储、物流、交易损耗等生产后环节的成本与收益。

    \item 在规划期内,所有种植活动所需的人力、设备、技术及其他生产资料均能得到充分供应,不构成生产决策的限制因素。

\end{enumerate}


\section{符号说明}

\begin{table}[H]
	\centering
	\caption{符号说明}
	\begin{tabular}{ll}
		\toprule
		符号                 & 说明                                \\
		\midrule

		$i, I$             & 地块的索引与集合                          \\
		$j, J$             & 作物的索引与集合                          \\
		$k, K$             & 季节的索引与集合                          \\
		$y, Y$             & 年份的索引与集合(2024-2030)               \\
		$J_{\text{bean}}$  & 所有豆类作物的集合                         \\

		$A_i$              & 地块$i$ 的可用面积                       \\
		$C_j$              & 作物$j$ 的单位面积种植成本                   \\
		$P_j$              & 作物$j$ 的单位重量销售价格                   \\
		$\text{Yield}_j$   & 作物$j$ 的单位面积产量                     \\
		$\text{Demand}_j$  & 作物$j$ 的每季预期市场销售量                  \\
		$\text{Past}_{ij}$ & 地块$i$ 在2023年是否种植了作物 $j$ 的0-1参数    \\
		$S_{ijk}$          & 地块$i$ 在季节 $k$ 是否适宜种植作物 $j$ 的0-1参数 \\
		$A_{\min}$         & 单个地块上允许种植某种作物的最小面积阈值              \\
		$N_j$              & 作物$j$ 在单季内允许种植的最大分散地块数量           \\
		$M$                & 大M方法中的一个足够大的正数                    \\
		\bottomrule
	\end{tabular}
\end{table}
\section{数据预处理与分析}


在构建优化模型之前,对原始数据进行预处理和分析是保证模型准确性和有效性的基础。本章对附件中提供的乡村耕地信息、作物种植特性及2023年相关统计数据进行清洗、整理和分析,为后续种植策略优化提供可靠的数据支撑。处理流程包括数据清洗与缺失值补全、土地资源分布统计、作物适宜性分析以及经济效益评估,其整体框架如图\ref{fig:data_process}所示。


% \begin{figure}[htbp]
%     \centering
%     \includegraphics[width=0.8\textwidth]{figures/data_process.png}
%     \caption{数据预处理与分析框架}
%     \label{fig:data_process}
% \end{figure}

\subsubsection{数据清洗与补全}

数据预处理的首要目标是构建完整且准确的2023年基准数据集。在整理过程中,我们发现智慧大棚第一季度的种植记录缺失。根据附件2中说明“2023年智慧大棚第一季度的数据与普通大棚相同”,我们采用确定性参照填充方法,将普通大棚同期的完整种植数据用于填充智慧大棚缺失字段,从而形成结构完整的数据集。

获得完整数据后,我们对其进行了全面质量评估以确保准确性。针对亩产量、种植成本及销售价格等数值型变量,我们计算了其均值、标准差与值域范围等描述性统计量,用以审查是否存在超出农业生产与市场常规的极端值。对于地块编号、作物类型等分类型变量,我们通过生成其频率分布表,以检验记录的一致性与有效性。评估结果显示,所有数据均在合理区间内,无异常值或逻辑错误,证明该数据集具有高度完整性和可靠性,可直接用于后续建模分析。



\input{contents/3问题一}
\input{contents/4问题二}
\input{contents/5问题三}
\section{灵敏度分析}

为检验模型在关键参数变动下的稳定性和可靠性,本章对模型中涉及的重要参数进行系统的灵敏度分析。分析主要围绕三个维度展开:贯穿所有模型的管理便利性约束参数,问题一中的核心资源要素,问题二中的决策者风险偏好,以及问题三中的市场反馈机制假设。

\subsection{分散性约束参数的灵敏度分析}

在模型的构建过程中,为量化田间管理的便利性,引入了两个关键的超参数:地块内允许种植的作物种类上限$p$,以及单一作物允许跨地块类型种植的数量上限$q$。这两个参数共同作用,在追求经济效益最大化的同时,避免了种植方案在空间上的过度零散化。为探究这两个参数对不同模型优化结果的影响,我们以基准值为中心,调整其取值,并观察七年规划期内总利润的变化情况。

分析结果如表\ref{tab:pq_sensitivity}所示。数据显示,对于问题一、二、三的优化模型,总利润均随参数$p$和$q$的取值增大而呈现上升趋势。这是因为增大$p$和$q$的取值,相当于放宽了模型的约束,使得算法拥有更大的可行解空间去寻找更高利润的组合。从变化率的绝对值来看,在所有模型中,参数$p$的变动对总利润的影响均大于参数$q$。例如,在问题一的模型中,将$p$从3收紧至2,利润下降2.53\%;而将$q$从4收紧至3,利润仅下降0.89\%。这表明,地块内作物品种的多样性是比作物跨地块类型分布更为敏感的因素。

此外,比较不同模型对参数变化的响应可以发现,问题二的鲁棒优化模型在参数$p$放宽时获得的利润增幅最大,为1.20\%,这可能是因为更大的灵活性有助于模型配置更多样化的种植组合以对冲不确定性风险。综合考虑利润提升与实际管理的可行性,我们认为在模型中选择的基准参数是合理的。

\begin{table}[H]
    \centering
    \caption{分散性约束参数$p, q$对各模型总利润的影响}
    \label{tab:pq_sensitivity}
    \begin{tabular}{@{}lcccc@{}}
        \toprule
        模型         & 参数 & 取值 & 七年总利润 (百万元) & 利润变化率 (\%) \\
        \midrule
        \multirow{5}{*}{问题一 (情景二)} 
                     & \multirow{3}{*}{$p$} & 2    & 40.85 & -2.53\% \\
                     &      & \textbf{3 (基准)} & \textbf{41.91} & \textbf{0.00\%}  \\
                     &      & 4    & 42.35 & 1.04\%  \\
        \cmidrule(l){2-5}
                     & \multirow{3}{*}{$q$} & 3    & 41.54 & -0.89\% \\
                     &      & \textbf{4 (基准)} & \textbf{41.91} & \textbf{0.00\%}  \\
                     &      & 5    & 42.12 & 0.49\%  \\
        \midrule
        \multirow{5}{*}{问题二 (鲁棒优化)}
                     & \multirow{3}{*}{$p$} & 2    & 43.52 & -1.43\% \\
                     &      & \textbf{3 (基准)} & \textbf{44.15} & \textbf{0.00\%}  \\
                     &      & 4    & 44.68 & 1.20\%  \\
        \cmidrule(l){2-5}
                     & \multirow{3}{*}{$q$} & 3    & 43.88 & -0.61\% \\
                     &      & \textbf{4 (基准)} & \textbf{44.15} & \textbf{0.00\%}  \\
                     &      & 5    & 44.31 & 0.36\%  \\
        \midrule
        \multirow{5}{*}{问题三 (动态反馈)}
                     & \multirow{3}{*}{$p$} & 2    & 38.56 & -1.18\% \\
                     &      & \textbf{3 (基准)} & \textbf{39.02} & \textbf{0.00\%}  \\
                     &      & 4    & 39.33 & 0.79\%  \\
        \cmidrule(l){2-5}
                     & \multirow{3}{*}{$q$} & 3    & 38.81 & -0.54\% \\
                     &      & \textbf{4 (基准)} & \textbf{39.02} & \textbf{0.00\%}  \\
                     &      & 5    & 39.15 & 0.33\%  \\
        \bottomrule
    \end{tabular}
\end{table}


\subsection{土地资源面积的灵敏度分析}

土地是农业生产最基本的资源要素,其总面积直接决定了生产规模的上限。本节针对问题一中情景二的确定性模型,分析七年总利润对土地总面积变化的响应程度。我们以附件中给定的总土地面积为基准,分别计算当总面积在-10\%至+10\%的区间内变动时,最优总利润的变化情况。

如图\ref{fig:land_sensitivity}所示,七年总利润与土地总面积的变化率之间呈现出一种强相关的正向关系。在所分析的变动范围内,总利润随土地面积的增加而近似线性增长。这一结果表明,在当前的资源和技术条件下,土地资源仍是限制该乡村农业总产出的关键因素,所制定的种植方案能够有效利用新增的土地资源并将其转化为经济收益,未出现明显的规模报酬递减现象。

\begin{figure}[H]
    \centering
    \includegraphics[width=\textwidth]{figs/6灵敏度分析/第一问地块灵敏度圆弧图.png}
    \caption{土地总面积变化对问题一最优总利润的影响}
    \label{fig:land_sensitivity}
\end{figure}


\subsection{风险偏好系数的灵敏度分析}

在问题二的鲁棒优化模型中,决策者的风险偏好是影响最终方案选择的重要因素。为系统评估不同风险偏好下的方案特征,我们引入风险偏好系数$a$,并选取了$a \in \{0.1, 0.3, 0.5, 0.7, 0.9\}$五个代表性水平进行独立优化。每个水平对应一个在有效前沿上的候选方案。表\ref{tab:risk_sensitivity_results}汇总了这些方案在期望利润、最低利润、风险(以利润标准差衡量)以及夏普比率\cite{SDCY202202002}四个核心指标上的表现。

\begin{table}[H]
    \centering
    \caption{不同风险偏好下的方案性能指标}
    \label{tab:risk_sensitivity_results}
    \begin{tabular}{@{}lcccc@{}}
        \toprule
        风险系数 $a$ & 期望利润 (百万元) & 最低利润 (百万元) & 风险 (标准差, 百万元) & 夏普比率 \\
        \midrule
        0.10 & 41.50 & 40.35 & 0.77 & 53.90 \\
        0.30 & 42.75 & 41.50 & 0.83 & 51.51 \\
        0.50 & 43.60 & 42.40 & 0.80 & 54.50 \\
        \textbf{0.70} & \textbf{44.15} & \textbf{43.10} & \textbf{0.70} & \textbf{63.07} \\
        0.90 & 44.80 & 43.50 & 0.87 & 51.49 \\
        \bottomrule
    \end{tabular}
\end{table}

数据显示,$a=0.7$对应的方案在各项指标中取得了最佳的综合平衡。该方案的夏普比率达到63.07,在所有候选中最高,表明其单位风险所能换取的超额回报最高。同时,该方案的风险水平(标准差0.70百万元)为所有候选中最低,而其期望利润(44.15百万元)和最低利润(43.10百万元)均处于高位。图\ref{fig:risk_preference_comparison}也直观展示了各方案在不同指标下的表现。基于此综合评估,我们将$a=0.7$的方案确定为问题二的最终推荐方案。

\begin{figure}[H]
    \centering
    \includegraphics[width=\textwidth]{figs/6灵敏度分析/第二问风险灵敏度.png}
    \caption{不同风险偏好系数下各性能指标的可视化对比。}
    \label{fig:risk_preference_comparison}
\end{figure}

\subsection{市场敏感度系数的灵敏度分析}

问题三的动态反馈模型建立在对市场响应机制的假设之上,其中价格与成本敏感度系数是定义该机制的核心参数。为检验最终优化结果对这些假设参数的稳健性,我们对设定的六个市场敏感度系数,即三类作物的价格敏感度与成本敏感度,进行了单因素灵敏度分析。在分析中,每次只变动一个参数,在其基准值附近取四个不同的水平,并为每个水平重新运行完整的优化过程,记录最终预期平均利润的变化。

分析结果如图\ref{fig:market_sensitivity}所示。可以观察到,预期总利润对食用菌的价格敏感度$p_{\text{fungi}}$变化最为敏感,其对应的曲线斜率绝对值最大。这表明,食用菌这类高价值经济作物的市场供需关系是影响该乡村总体经济效益的关键因素,对这类作物的产量规划需要更为审慎。相比之下,总利润对粮食和蔬菜的价格及成本敏感度表现出较强的不敏感性,即使这些参数发生一定范围的变化,最优利润水平也基本保持稳定。这说明,对于大宗农产品,本模型给出的种植策略具有较好的适用性和稳定性。

\begin{figure}[H]
    \centering
    \includegraphics[width=\textwidth]{figs/6灵敏度分析/问题三灵敏度分析图.png}
    \caption{预期总利润对六个核心敏感度系数的响应分析。}
    \label{fig:market_sensitivity}
\end{figure}










\newpage

% 参考文献
\bibliographystyle{plain}
\bibliography{reference}
\newpage

\end{document}