% \documentclass{cumcmthesis}
\documentclass[withoutpreface,bwprint]{cumcmthesis} %去掉封面与编号页,电子版提交的时候使用。


\usepackage[noend]{algpseudocode}
\usepackage{algorithm}
\usepackage{algpseudocode}
\usepackage{amsmath}
\usepackage[framemethod=TikZ]{mdframed}
\usepackage{url}   % 网页链接
\usepackage{subcaption} % 子标题
\usepackage{threeparttable} 
\title{面向不确定性与系统复杂性的农作物种植策略优化研究}
\tihao{A}
\baominghao{4321}
\schoolname{XX大学}
\membera{ }
\memberb{ }
\memberc{ }
\supervisor{ }%辅导老师
\yearinput{2023}
\monthinput{9}
\dayinput{8}

\setcounter{tocdepth}{2}

\begin{document}

\maketitle
\begin{abstract}
在乡村现代化与土地资源高效利用的背景下,本研究聚焦于农业生产中的资源配置优化问题。针对华北某山区乡村,旨在制定一个覆盖2024至2030年的最优农作物种植方案。该乡村的生产决策面临土地异质性、多重农艺规则以及田间管理便利性等一系列硬性约束,构成了一个复杂的约束优化环境。为应对此挑战,本文通过递进式的建模方法,旨在为该乡村提供兼顾经济效益最大化与风险稳健性的数据驱动型决策支持。

针对问题一,我们在所有经济与生产参数保持不变的确定性假设下,构建了一个多周期混合整数线性规划(MILP)模型。该模型遵循微观经济学中厂商利润最大化原则,在给定的生产技术与资源下,求解最优生产组合。模型分别探讨了当边际产出超出市场预期时,产品滞销或以折价进入次级市场的两种情景。鉴于该组合优化问题的计算复杂性,我们设计并实现了一种修复式遗传算法进行求解。结果表明,在滞销情景下,七年最优总利润为2987.65万元;在降价出售情景下,该值可提升至4191.35万元。

针对问题二,我们放宽了确定性假设,引入了市场与生产环境中的不确定性。模型将亩产量、预期销售量和销售价格等关键参数处理为在预设区间内波动的随机变量。为管理由此产生的风险,我们建立了一个鲁棒优化模型,其目标从最大化期望利润转变为最大化风险规避下的年度平均保底利润。通过采用多种群遗传算法(MPGA)求解,并基于风险调整后收益(夏普比率)进行方案选择,最终确定了夏普比率最高(63.07)的种植方案。该方案在确保4310.00万元最低利润的同时,实现了4415.00万元的期望利润,获得了更优的风险-收益权衡。

针对问题三,我们在模型中进一步引入了市场动态反馈机制,以反映供给变化对市场均衡价格与要素成本的影响。通过构建基于价格与成本敏感度系数的仿真优化模型,我们分析了乡村作为市场参与者的行为对自身经济环境的反作用。模型继续使用多种群遗传算法,在集成了蒙特卡洛模拟的适应度评估框架下进行求解。最终得到的自适应种植策略,其预期七年平均总利润为3902.05万元。利润分布的统计分析显示,方案收益高度稳定,有95\%的概率实现不低于3865.18万元的总利润,验证了该策略在动态市场环境下的稳健性。

综上所述,本研究通过构建从静态确定性规划、风险规避下的鲁棒优化到动态仿真优化的递进式模型体系,为该乡村制定了科学、详尽且可行的长期种植方案。研究成果不仅为具体的生产活动提供了指导,也证明了系统建模方法在解决现实农业经济管理问题中的有效性。最终方案在多重约束和不确定性下,实现了3902.05万元的稳定预期总利润,为促进区域农业经济的可持续发展提供了决策依据。

\keywords{混合整数线性规划 \quad 鲁棒优化 \quad 仿真优化 \quad 遗传算法 \quad 农业经济}


\end{abstract}








\section{问题背景}

在乡村振兴与农业现代化背景下,科学规划并高效利用有限土地,对保障区域粮食安全和促进乡村经济可持续发展至关重要。该问题涉及资源配置的运筹学建模,同时关联生态经济中的经济—生态系统相互作用和区域经济的内生增长机制。作物生产与品种选择受地域、气候和土壤等自然条件约束;其经济效益则依赖于市场需求、生产成本与销售价格等动态变量。基于此,需要制定兼顾经济产出最大化、生态稳定性与风险应对能力的长期种植策略\upcite{GZZP202402041}。

本研究以华北某山区乡村为对象。该乡村兼具露天耕地与设施大棚等多类型土地资源,并受轮作制度、豆类种植比例及田间管理可及性等多重约束。上述现实因素构成了一个多维约束的决策环境。为此,本文采用系统的数学建模方法,构建覆盖2024至2030年的最优种植方案,以期为土地高效利用与经济稳健增长提供数据驱动的决策支持。


\section{问题重述}

问题一:在所有经济与生产参数取基准值且保持不变的条件下,针对 2024–2030 年规划期,求解最优种植结构以最大化七年累计利润。分别在两种需求情景下求解:(i)产量超出预期的部分无法销售;(ii)超出部分以基准价格的 50\% 出售。

问题二:在问题一框架上引入参数不确定性。考虑预期销售量、单位面积产量、单位面积种植成本和销售价格在给定区间内的波动,设计一个在整个规划期内保持不变的单一种植方案,以收益—风险权衡为目标,提升方案鲁棒性,并在不利市场与生产情景下保持经济绩效稳定。

问题三:在问题二基础上纳入作物间的替代性与互补性,以及预期销售量、销售价格与种植成本的相关结构。基于这些关系的建模与放好着呢,求解可随市场参数变化更新的自适应种植策略,并与问题二的鲁棒方案进行对比评估。


\section{问题分析}

对于问题一,其在确定性前提下提出:研究对象为多周期、多地块、多品类的资源分配问题,且假定所有生产与市场参数为已知且恒定,目标为在规划期内实现累计利润最大化。该设定与古典微观经济学中厂商理论的理性与利润最大化假设一致,因此在既定资源与农艺约束下的最优决策过程可被视为对单一目标决策主体的数学建模。由此,该问题可形式化为一个混合整数线性规划(MILP)模型,以便在约束条件下求解最优种植策略。

对于问题二,其本质上是将确定性优化推广到不确定情形:预期销量、单位产量、单位种植成本与销售价格作为在预设区间内波动的有界不确定量,且题设未给出其概率分布。因缺乏分布信息,基于概率分布的随机规划方法不可行,因此应采用仅依赖波动边界的建模策略,即鲁棒优化。鲁棒优化通过最大化最坏情形下的性能以体现风险规避倾向,相较于以期望收益为目标的方法更侧重最低收益保障与下行风险控制,从而在数据有限的条件下生成更为审慎且性能稳定的种植方案。

对于问题三,其在模型中引入作物间的替代性与互补性以及价格—需求反馈等系统性相互作用。由于这些相互作用产生显著的动态非线性,模型目标函数无法用封闭的解析表达式精确表示,故基于显式解析模型的优化方法难以适用。因而应将该问题视为黑箱优化问题,并采用仿真—优化框架:首先构建能够再现关键市场与生产动态的仿真器;其次采用不依赖问题解析形式的全局搜索算法在方案空间中寻优;最后通过仿真对所得策略的性能与稳健性进行评估。我们最终的框架如图\ref{fig:all}所示。


\begin{figure}[htbp]
	\centering
	\includegraphics[width=\textwidth]{figs/1前置/全文框架}
	\caption{整体框架}
	\label{fig:all}
\end{figure}


\section{模型假设}


为确保数学模型的有效性与针对性,并明确其适用边界,我们提出以下四条基本假设:

\begin{enumerate}
    \item 模型所使用的全部历史数据均视为准确可靠。2023年的产销数据被视为代表了市场的均衡状态。

    \item 该乡村在农产品市场中被视为价格接受者。其自身的任何生产决策所导致的供给量变化,均不足以对该作物的市场结算价格产生实质性影响。

    \item 模型的经济效益评估聚焦于种植环节。所有计划销售的农产品,无论是正常价格或降价销售,均假定在当季完成交易。模型不包含仓储、物流、交易损耗等生产后环节的成本与收益。

    \item 在规划期内,所有种植活动所需的人力、设备、技术及其他生产资料均能得到充分供应,不构成生产决策的限制因素。

\end{enumerate}


\section{符号说明}

\begin{table}[H]
	\centering
	\caption{符号说明}
	\begin{tabular}{ll}
		\toprule
		符号                 & 说明                                \\
		\midrule

		$i, I$             & 地块的索引与集合                          \\
		$j, J$             & 作物的索引与集合                          \\
		$k, K$             & 季节的索引与集合                          \\
		$y, Y$             & 年份的索引与集合(2024-2030)               \\
		$J_{\text{bean}}$  & 所有豆类作物的集合                         \\

		$A_i$              & 地块$i$ 的可用面积                       \\
		$C_j$              & 作物$j$ 的单位面积种植成本                   \\
		$P_j$              & 作物$j$ 的单位重量销售价格                   \\
		$\text{Yield}_j$   & 作物$j$ 的单位面积产量                     \\
		$\text{Demand}_j$  & 作物$j$ 的每季预期市场销售量                  \\
		$\text{Past}_{ij}$ & 地块$i$ 在2023年是否种植了作物 $j$ 的0-1参数    \\
		$S_{ijk}$          & 地块$i$ 在季节 $k$ 是否适宜种植作物 $j$ 的0-1参数 \\
		$A_{\min}$         & 单个地块上允许种植某种作物的最小面积阈值              \\
		$N_j$              & 作物$j$ 在单季内允许种植的最大分散地块数量           \\
		$M$                & 大M方法中的一个足够大的正数                    \\
		\bottomrule
	\end{tabular}
\end{table}
\section{数据预处理与分析}


在构建优化模型之前,对原始数据进行预处理和分析是保证模型准确性和有效性的基础。本章对附件中提供的乡村耕地信息、作物种植特性及2023年相关统计数据进行清洗、整理和分析,为后续种植策略优化提供可靠的数据支撑。处理流程包括数据清洗与缺失值补全、土地资源分布统计、作物适宜性分析以及经济效益评估,其整体框架如图\ref{fig:data_process}所示。


% \begin{figure}[htbp]
%     \centering
%     \includegraphics[width=0.8\textwidth]{figures/data_process.png}
%     \caption{数据预处理与分析框架}
%     \label{fig:data_process}
% \end{figure}

\subsubsection{数据清洗与补全}

数据预处理的首要目标是构建完整且准确的2023年基准数据集。在整理过程中,我们发现智慧大棚第一季度的种植记录缺失。根据附件2中说明“2023年智慧大棚第一季度的数据与普通大棚相同”,我们采用确定性参照填充方法,将普通大棚同期的完整种植数据用于填充智慧大棚缺失字段,从而形成结构完整的数据集。

获得完整数据后,我们对其进行了全面质量评估以确保准确性。针对亩产量、种植成本及销售价格等数值型变量,我们计算了其均值、标准差与值域范围等描述性统计量,用以审查是否存在超出农业生产与市场常规的极端值。对于地块编号、作物类型等分类型变量,我们通过生成其频率分布表,以检验记录的一致性与有效性。评估结果显示,所有数据均在合理区间内,无异常值或逻辑错误,证明该数据集具有高度完整性和可靠性,可直接用于后续建模分析。



\input{contents/3问题一}
\input{contents/4问题二}
\input{contents/5问题三}









\newpage

% 参考文献
\bibliographystyle{plain}
\bibliography{reference}
\newpage

\end{document}